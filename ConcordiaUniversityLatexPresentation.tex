\documentclass{beamer}

\usepackage{xcolor}
\usepackage{caption}
\usepackage{graphicx}
\usepackage{array}
\usepackage{amssymb} 
\usepackage{amsmath}
\usepackage{multimedia}
\usepackage{hyperref}
\usetheme{Madrid}
\graphicspath{ {Pictures/} }
\definecolor{lightblue}{RGB}{0,73,114}
\definecolor{grund}{RGB}{238,241,251}          
\definecolor{schrift}{RGB}{0,73,114}
\definecolor{magenta}{RGB}{128,0,128}

\definecolor{maroon}{RGB}{113, 41, 21  }
\setbeamercolor{structure}{fg=maroon}
\title[]{Concordia University}

% A subtitle is optional and this may be deleted
\subtitle{Concordia University Presentation}

\author[]{Name 1 \\ \and Name 2\\ \and Name 3\\ \and Name 4\\}  


% - Give the names in the same order as the appear in the paper.
% - Use the \inst{?} command only if the authors have different
%   affiliation.

\institute[] % (optional, but mostly needed)
{
%Change Team Name
  Concordia Team\\
}
% - Use the \inst command only if there are several affiliations.
% - Keep it simple, no one is interested in your street address.

\date{}
% - Either use conference name or its abbreviation.
% - Not really informative to the audience, more for people (including
%   yourself) who are reading the slides online

\subject{Concordia University}
% This is only inserted into the PDF information catalog. Can be left
% out. 

% If you have a file called "university-logo-filename.xxx", where xxx
% is a graphic format that can be processed by latex or pdflatex,
% resp., then you can add a logo as follows:

\pgfdeclareimage[height=1cm]{university-logo}{./Pictures/university-logo-Concordia}
\logo{\pgfuseimage{university-logo}}




% - Either use conference name or its abbreviation.
% - Not really informative to the audience, more for people (including
%   yourself) who are reading the slides online

\subject{SOEN 6441, Advanced Programming Practices}

\AtBeginSection[]
{
  \begin{frame}<beamer>{Contents of the section}
    \tableofcontents[currentsection]
    %,currentsubsection]
  \end{frame}
}

% Let's get started
\begin{document}

\begin{frame}
\titlepage
\end{frame}

\begin{frame}{Outline}
  \tableofcontents
  % You might wish to add the option [pausesections]
\end{frame}

% Section and subsections will appear in the presentation overview
% and table of contents.
\section{Project Description}
\begin{frame}{Project Description}

\end{frame}

%\subsection{Second Subsection}

% You can reveal the parts of a slide one at a time
% with the \pause command:
\section{Content}
\begin{frame}{Content}

\begin{block}{Block}

Add text here ! 

\end{block}
\end{frame}

\begin{frame}{Continuing Content Section}

\begin{block}{Content}

Block

\end{block}

\end{frame}

\begin{frame}{Continuing Content Section}

\textbf{List}

\begin{itemize}
  \item {
   1
    %\pause % The slide will pause after showing the first item
  }
   \item {
    2
    %\pause % The slide will pause after showing the first item
  }
  \item {
    3
    %\pause % The slide will pause after showing the first item
  }
  \end{itemize}
  
\begin{block}{Block}
  Block
\end{block}

\end{frame}

\section{Another Section}
\begin{frame}{Another Section}

 
  
\end{frame}



\section{Tools}
\begin{frame}{Tools}
  
\end{frame}


% Placing a * after \section means it will not show in the
% outline or table of contents.

\section{Conclusion}

\begin{frame}{Conclusion}
 

\end{frame}

\section{Q\&A}
\begin{frame}{Q\&A}
\begin{figure}
\includegraphics[scale=0.5]{qa}
\end{figure}
\end{frame}



% All of the following is optional and typically not needed. 
\appendix
\section<presentation>*{\appendixname}
\subsection<presentation>*{References}

\begin{frame}[allowframebreaks]
  \frametitle<presentation>{For Further Reading}
    
  \begin{thebibliography}{10}
    
  \beamertemplatebookbibitems
  % Start with overview books.

  \bibitem{Author1990}
    A.~Author.
    \newblock {\em Handbook of Everything}.
    \newblock Some Press, 1990.
 
    
  \beamertemplatearticlebibitems
  % Followed by interesting articles. Keep the list short. 

  \bibitem{Someone2000}
    S.~Someone.
    \newblock On this and that.
    \newblock {\em Journal of This and That}, 2(1):50--100,
    2000.
  \end{thebibliography}
\end{frame}

\end{document}


